\chapter{Conclusions and Future Work}

\section{Summary of Provided Solution}

As can be seen from the diffs of Warranty Files from the various extractions, replayed data is consistent with
traditionally extracted data in all respects, and is in fact more consistent in some areas.

\subsection{Forensic Soundness}


The NIJ and McKemmish both address forensic soundness, though their approaches are similar enough to be
treated together.

The NIJ's definition of forensic soundness has three elements:

\begin{enumerate}
  \item Any process or procedure of collecting, transporting, or storing of digital evidence should not incur any changes to the evidence.
  \item Only specifically trained experts should examine digital evidence.
  \item Transparency during the operations of acquisition, transportation, and storage of the evidence should be maintained.
\end{enumerate}

McKemmish's similar definition has four:

\begin{itemize}
\item \emph{Meaning} is a term that denotes confidence in the interpretation of extracted evidence data.
\item \emph{Error Detection} denotes processes for detecting or predicting errors in the forensic process.
\item \emph{Transparency} means the forensic process is documented, known, and verifiable.
\item \emph{Expertise} is required for those investigators examining digital data.
\end{itemize}

As can be seen from the verification results, the original evidence is modified less using forensic replay than it is by repeatedly
extracting the information by traditional means. Even if no additional faults are created during a bench download, the ECM running
time is better preserved by imaging and replaying the ECM data rather than repeated downloads.

The expertise requirement is not totally alleviated by the extraction and replay process; an investigator still has to know how to 
connect to the ECM and power it up properly. However, the main advantage of the forensic extraction and replay process is that
all information extraction is automated; no knowledge of diagnostic software or the steps required to gather pertinent crash information
is needed. This is an advantage in a law enforcement context where training time is at a premium.

By default, the CAT ECM extraction process is almost completely opaque. Unless certain data are not available on the ECM, or an error
occurs during the download process, there is no record kept of the traffic other than its final interpretation by the maintenance
software. As the forensic replay method is literally a recording of network traffic, this traffic can be examined after the fact
to verify the extraction and replay process.

Eoghan Casey's definition of forensic soundness includes seven levels of forensic certainty based on certain characteristics
of the evidence and the method in which it is stored:

\begin{itemize}
\item C0: Evidence contradicts known facts
\item C1: Evidence is highly questionable.
\item C2: Only one source of evidence that is not protected against tampering
\item C3: Source(s) of evidence are more difficult to tamper with but there is insufficient evidence for a firm conclusion, or unexplained inconsistencies exist in available evidence
\item C4: Sole source evidence is protected against tampering or multiple, independent sources of evidence agree but the independent evidence is unprotected from tampering.
\item C5: Agreement of evidence from multiple, independent sources protected from tampering, but small uncertainties exist.
\item C6: The evidence is tamper proof and unquestionable.
\end{itemize}

The solution outlined herein makes no claim about the veracity of the evidence as it relates to physical phenomena, except that it is a faithful representation of the information communicated
by the ECM, but the other characteristics of evidence can be compared. The main difference here is that when extracting and storing evidence through traditional means, even if the
evidence agrees with other sources it is not tamper-proof, and thus can only achieve a rating of C4 in the absolute best case (i.e. there is physical evidence corroborating the 
digital evidence). However, with the tamper protections added by the cryptosystem presented herein, the digital evidence is rated at C4 at bare minimum, reaching C5 or C6 with corroborating
physical evidence. 

\subsection{Practicality}

Currently, in order to perform an evidence download from a heavy vehicle ECM, investigators require at bare minimum a laptop and a diagnostic link connector.
If a bench download is to be performed, an engine flashing harness is required as well. Unfortunately, removing the need for an engine flashing harness is
beyond the scope of this work. However, this replay solution represents a logistical advantage over existing methods.

While the current use case still requires a laptop, it does not require any software beyond a simple web browser. This greatly simplifies software logistics, as well
as reducing the cost of software required to conduct investigations in the field. The hardware can be made no larger or heavier than existing diagnostic link connectors
while remaining far cheaper.

\section{Future Work}

\subsection{Understanding Meaning}

The current solution requires that investigators use the manufacturers' maintenance software to interpret the data that has been extracted and replayed from the ECM.
While this is an improvement over the current situation in terms of ease and practicality, it is also not ideal. Future work in this area should include analyzing the
meaning of the extracted ECM data, so that it can be analyzed and presented independently of proprietary software.

\subsection{Information Verification}

New work might also include verification of the interpretation of the data. Currently, it is assumed that the data reported by the ECM accurately reflect the conditions
observed during incidents, and that the manufacturers' software accurately interprets this reported data. 

% LocalWords:  ics
