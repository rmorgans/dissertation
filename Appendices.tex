%\begin{appendices}
\TUappendix{Code Listings}\label{app:code}

\TUsection{Extraction Code}

\TUsubsection{catextract.py}\label{app:extraction}
%\lstset{basicstyle=\small}
%\lstinputlisting[language=Python]{catreplay/simple-extract.py}
\begin{minted}{python}
import RP1210
import logfileparser
import sys
from RP1210Defines import *
import catATA
import time
import ATAClient
import atamessage
import pickle
import J1587_Transport as transport
import responses as response_utils
#import threading


excluded_pids = [[0xfc,0x58]]
multiple_pids = {0xDB:8,0xD3:10}


dupefile = open("dupefile.txt","w")
def get_proprietary(driver):
   start = time.time()
   while True:
      if time.time() - start > 2:
         print("timed out while waiting for response")
         return None
      msg = driver.read_message()
      if len(msg) < 5:
         continue
      if catATA.is_proprietary(msg) or msg[1] == 0xc5:
#         print(msg[4:])
         return msg
      else:
         continue

def get_num_responses(pid):
   if pid in excluded_pids:
      return 0
   elif pid in multiple_pids.keys():
      return multiple_pids[pid]
   else:
      return 1

def get_pid_msg(msg,key,driver):
   got_pid = False
   this_pid = msg.pid
   start = time.time()
   retmsg = None
   while (not got_pid) and (not time.time() - start > 5):
      inbuf = get_proprietary(driver)
      if inbuf is None:
         #driver.send_message(msg.to_buffer())
         continue

      elif inbuf[1] == 0xc5:
         print("transport protocol engage")
         tmp = transport.receive_message(msg.src,msg.dst,driver,rts=inbuf)#check these src dst assignments if there's problems
         if tmp[0] == 0xfe:
            retmsg = atamessage.ATAMessage.from_transport(msg.dst,msg.src,tmp,key=key)
            print("I think the PID is %d, hoping for %d" % (retmsg.pid,this_pid))
      else:
         retmsg = atamessage.ATAMessage.from_buf(inbuf,key=key)
         print("I think the PID is %d, hoping for %d" % (retmsg.pid,this_pid))

      if retmsg.pid == this_pid:
         got_pid = True
      else:
         retmsg = None

   return retmsg



def get_pid(msg,key,driver):
   if msg.data in excluded_pids:
      return []
   pid = msg.pid
   responses = []
   numresponses = get_num_responses(pid)
   i = 0
   tries = 0
   if numresponses > 1:
      dupefile.write("\n\n")
   while i < numresponses:
      respmsg = get_pid_msg(msg,key,driver)

      if respmsg is not None:
         dupe = False
         for r in responses:
            if response_utils.compare_J1708_msg(respmsg.to_plain_buffer(),r.to_plain_buffer()):
               dupe = True
               break
         if dupe:
            continue
         responses.append(respmsg)
         i += 1
      else:
         i = 0
         driver.send_message(msg.to_buffer())
   if len(responses) > 1:
      for r in responses:
         dupefile.write(str(r.to_plain_buffer()))
   return responses





def get_setup(msg,driver):
   setup = False
   while not setup:
      inmsg = get_proprietary(j1708)
      if inmsg is None:
         driver.send_message(msg)
         continue
      if inmsg[1] == 0xc5:
         continue
      thismsg = atamessage.ATAMessage.from_buf(inmsg)
      if thismsg.is_setup():
         return thismsg.data[-1] & 0x0F
      else:
         continue


def get_encrypted_response(driver):
   return get_command(driver,[0xB0,0xC0])

def get_unencrypted_response(driver):
   return get_command(driver,[0x90])

def get_encryption_setup(driver):
   return get_command(driver,[0xF0])


   


driver = RP1210.RP1210Driver("DPA4PMA.DLL")
driver.initialize()


try:
   fh = open(sys.argv[1],"r")
except:
   print(sys.argv[1])
   print("error opening file")
   sys.exit(-1)
   
tokens = logfileparser.tokenize_file(fh)

   
tdriver = RP1210.RP1210Driver("DPA4PMA.DLL")
tdriver.initialize()
j1708 = RP1210.J1708_Provider("DPA4PMA.DLL",driver=tdriver)
j1939 = RP1210.J1939_Provider("DPA4PMA.DLL",driver=tdriver)
j1708.initialize()
j1939.initialize()

print("all set up, press any key to start extracting.")

sys.stdin.read(1)
ataclient = ATAClient.ATAClient(j1708)
protocols = {}
J1708ResponseQueue = []
J1939ResponseQueue = []
for i in range(len(tokens)):
   token = tokens[i]
   buf = token.split(",")
   
   if buf[1] == "SC" or buf[1] == "CD":
      continue
   elif buf[1] == "CC" and buf[2] == "00":
      protocols[buf[3]] = buf[5]
   elif buf[1] == "SM":
      #print(buf)
      msg = logfileparser.send_message(token,proto=protocols[buf[0]])
      if protocols[buf[0]] == "J1708":
         if len(msg.data) < 5 or not msg.data[1] == 0xFE:
            continue
         j1708.send_message(msg.data)
         sm = atamessage.ATAMessage.from_buf(msg.data,key=ataclient.encrypt_key)
         if sm.is_setup():
            ataclient.encrypt_key = sm.data[-1] & 0x0F
            ataclient.decrypt_key = get_setup(msg.data,j1708)
            print("encrypt key: %d decrypt key: %d" % (ataclient.encrypt_key,ataclient.decrypt_key))
         else:
            print("Trying to match %s" % sm.to_plain_buffer())
            responses = get_pid(sm,ataclient.decrypt_key,j1708)
            if len(responses) > 0:
               thisqueue = []
               for response in responses:
                  thisqueue.append(response.to_plain_buffer())
               J1708ResponseQueue.append([sm.to_plain_buffer(),thisqueue])
               print(J1708ResponseQueue[-1])

         time.sleep(.3)

         
      elif protocols[buf[0]] == "J1939":
         #msg = logfileparser.send_message(token,proto=protocols[buf[0]])
         j1939.send_raw(msg.data)
#         time.sleep(.3)
#         print("j1939 sent: %s" % msg.data)
         inmsg = j1939.read_raw()
         while len(inmsg) == 0:
            j1939.send_raw(msg.data)
            inmsg = j1939.read_raw()

         J1939ResponseQueue.append([msg.data,inmsg])
         print(J1939ResponseQueue[-1])
         time.sleep(.3)

      
      #driver.send_message(msg.client_id,msg.data)
      #just send this message.
   else:
      pass

del(j1708)
del(j1939)
fh.close()

#fh2 = open("responses2.txt","w")

#for response in ataclient.J1708responses:
#   fh2.write("%s   %s\n" % (list(map(hex,response[0].to_plain_buffer())),list(map(hex,response[1].to_plain_buffer()))))

#fh2.close()

try:
   queues = (J1708ResponseQueue,J1939ResponseQueue)
   response_queues = open("responsequeues.pkl","wb")
   pickle.dump(queues,response_queues)
   response_queues.close()
except:
   print("Problem pickling responses")


print("Extraction complete!\a\a\a\a")
\end{minted}

\TUsubsection{catreplay.py}\label{app:replay}
\begin{minted}{python}
import RP1210
import logfileparser
import sys
from RP1210Defines import *
import catATA
import time
import ATAClient
import atamessage
import pickle
from multiprocessing import Process
from responses import ResponseQueue
import J1587_Transport as transport
#import threading

encrypt_key = None
decrypt_key = None

'''tdriver = RP1210.RP1210Driver("DPA4PMA.DLL")
tdriver.initialize()

j1939 = RP1210.J1939_Provider("DPA4PMA.DLL",driver=tdriver)
j1708.initialize()
j1939.initialize()'''

def j1708_replay_callback(recvd_msg,resp_queue):
  global encrypt_key
  global decrypt_key
  if not atamessage.is_proprietary(recvd_msg):
    return
  if len(recvd_msg) < 5:
    return
  inmessage = atamessage.ATAMessage.from_buf(recvd_msg,key=decrypt_key)
  to_send = []
  if inmessage.is_setup():
    #print("setting decrypt key to %d" % (inmessage.data[-1] & 0x0F))
    decrypt_key = inmessage.data[-1] & 0x0F
    to_send.append(atamessage.ATAMessage(0x80,0xAC,atamessage.SECURITY_SETUP,[0x0B]))
    encrypt_key = 0x0B & 0x0F
    return to_send
  else:
    #print("Received: %s" % list(map(hex,inmessage.to_plain_buffer())))
    to_send_raw = resp_queue.get_j1708(inmessage.to_plain_buffer())
    if to_send_raw is not None and len(to_send_raw) > 0:
      for tsr in to_send_raw:
        to_send.append(atamessage.ATAMessage.from_plain_buf(tsr,key=encrypt_key))
        if len(to_send_raw) > 1:
          print("Sending: %s" % list(map(hex,tsr)))
      return to_send
    else:
      return None

def j1939_replay_callback(recvd_msg,resp_queue):
  outmsg = resp_queue.get_j1939(recvd_msg)
  return outmsg

def j1708_replay_proc(resp_queue):
#  tdriver = RP1210.RP1210Driver("DPA4PMA.DLL")
#  tdriver.initialize()
  driver = RP1210.J1708_Provider("DPA4PMA.DLL")
  driver.initialize()
  while True:
    inmsg = driver.read_message()
    outmsg_list = j1708_replay_callback(inmsg,resp_queue)
    if outmsg_list is None or len(outmsg_list) == 0:
      print("no outmsgs, continuing")
      continue
    else:
      for outmsg in outmsg_list:
        #print("sending %s" % outmsg)
        time.sleep(.05)
        if len(outmsg.data) + 2 <= 21:
          driver.send_message(outmsg.to_buffer())
        else:
          (src,dst,data) = outmsg.to_transport()
          transport.send_message(src,dst,data,driver)


def j1939_replay_proc(resp_queue):
  #tdriver = RP1210.RP1210Driver("DPA4PMA.DLL")
  #tdriver.initialize()
  driver = RP1210.J1939_Provider("DPA4PMA.DLL")
  driver.initialize()
  while True:
    inmsg = driver.read_raw()
    print("J1939 Received: %s" % list(map(hex,inmsg)))
    outmsg = j1939_replay_callback(inmsg,resp_queue)
    if outmsg:
      driver.send_raw(outmsg)


if __name__ == '__main__':
  try:
    queuesfh = open("responsequeues.pkl","rb")
    queues = pickle.load(queuesfh)
  except:
    print("Could not load response queues.")
    sys.exit(1)

  j1708_queue = queues[0]
  
  j1939_queue = queues[1]

  resp_queue = ResponseQueue(J1708requests=j1708_queue,J1939requests=j1939_queue)

  print("all set up, press any key to start replay.")
  sys.stdin.read(1)

  j1708process = Process(target=j1708_replay_proc,args=(resp_queue,))
  j1939process = Process(target=j1939_replay_proc,args=(resp_queue,))

  j1708process.start()
  j1939process.start()

  while True:
    q = sys.stdin.read(1)
    if q == "q":
      j1708process.terminate()
      j1939process.terminate()
      sys.exit(1)




\end{minted}

\TUsubsection{logfileparser.py}
\begin{minted}{python}
import re
import sys


#matches the beginning of a message but not a wrapped-around message
token = re.compile('(\d\d|XX),(CC|CD|SM|SC|RM).*')
fragment = re.compile('(\S\S,?)+$')
messages = []

class rp1210_client():
  def __init__(self,client_id,flags):
    self.client_id = client_id
    self.flags = flags

def tokenize_file(fh):
  lines = fh.readlines()
  tokens = []
  for line in lines:
    mymatch = token.search(line)
    if mymatch:
      tokens.append(mymatch.group(0))
    else:
      newmatch = fragment.search(line)
      tokens[-1] += newmatch.group(0)
  
  for i in range(len(tokens)):
    if tokens[i][-1] == ",":
      tokens[i] = tokens[i][:-1]
      
  return tokens
  
class rp1210_msg():
  def __init__(self,tokenstring):
    self.dat = tokenstring.split(",")
    self.message_type = self.dat[1]
    
class client_connect(rp1210_msg):
  def __init__(self,tokenstring):
    super().__init__(tokenstring)
    self.client_id = int(self.dat[3])
    self.protocol = self.dat[5]
    #This should be enough. I don't think we need to know Tx/Rx buffer sizes 
    
class send_command(rp1210_msg):
  def __init__(self,tokenstring):
    super().__init__(tokenstring)
    self.client_id = int(self.dat[0])
    self.retval = int(self.dat[2])
    self.msg_size = int(self.dat[3])
    self.command_number = int(self.dat[4])
    if self.msg_size > 0:
      self.msg_string = list(map(lambda x: chr(int(x,16)),self.dat[5:]))
    else:
      self.msg_string = ""
    
class send_message(rp1210_msg):
  def __init__(self,tokenstring,proto="J1939"):
    super().__init__(tokenstring)
    self.client_id = int(self.dat[0])
    self.bytes_sent = int(self.dat[3])
    if proto == "J1708":#Account for the first byte in the buffer being the priority byte.
      self.data = list(map(lambda x: int(x,16),self.dat[7:]))
    else:
      self.data = list(map(lambda x: int(x,16),self.dat[6:]))
    
#note: track client_connect calls to see if need to account for echo byte
class read_message(rp1210_msg):
  def __init__(self,tokenstring):
    super().__init__(tokenstring)
    self.client_id = int(self.dat[0])
    self.bytes_received = int(self.dat[2])
    self.timestamp = int("".join(self.dat[5:9]),16)
    self.data = list(map(lambda x: int(x,16),self.dat[9:]))
  
if __name__ == "__main__":
  filename = sys.argv[1]
  try:
    fh = open(filename,'r')
  except:
    print("problem reading file")
    sys.exit(1)
    
  tokenlist = tokenize_file(fh)
      
  for a in tokenlist:
    print(a)
    




\end{minted}

\TUsubsection{ATAClient.py}
\begin{minted}{python}

import atamessage
import J1587_Transport as transport
import copy


class j1939holder():
  def __init__(self,indata):
    self.data = indata
  def to_plain_buffer(self):
    return self.data

class ATAClient():
  def __init__(self,driver,J1708responses=None,J1939responses=None):
    self.driver = driver
    self.encrypt_key = None
    self.decrypt_key = None
    self.src = 0xAC
    if J1708responses is None:
      self.J1708responses = []
    if J1939responses is None:
      self.J1939responses = []

  #callback for when extractions are performed
  def extraction_callback(self,sent_msg,recvd_msgs):
    outmessage = atamessage.ATAMessage.from_buf(sent_msg,key=self.encrypt_key)
    print("Sent: %s" % list(map(hex,sent_msg)))
    this_queue = []
    for recvd_msg in recvd_msgs:
      if atamessage.is_proprietary(recvd_msg):
        print("Received: %s" % list(map(hex,recvd_msg)))
        inmessage = atamessage.ATAMessage.from_buf(recvd_msg,key=self.decrypt_key)
        if inmessage.is_setup():
          print(list(map(hex,recvd_msg)))
          print("setting decrypt key to %d" % (inmessage.data[-1] & 0x0F))
          self.decrypt_key = inmessage.data[-1] & 0x0F
        else:
          this_queue.append(inmessage.to_plain_buffer())
        #self.J1708responses.append((outmessage.to_plain_buffer(),inmessage.to_plain_buffer()))

      elif transport.is_rts(recvd_msg):
        src = recvd_msg[0]
        dst = recvd_msg[3]
        msg = None
        while msg is None:
          msg = transport.receive_message(self.src,dst,self.driver,recvd_msg)
          if msg is None:
            print("resending request")
          self.driver.send_message(outmessage.to_buffer())
        print("Received transport: %s" % msg)
        if msg[0] == 0xFE:
          inmessage = atamessage.ATAMessage.from_transport(src,dst,msg,key=self.decrypt_key)
          #self.J1708responses.append((outmessage.to_plain_buffer(),inmessage.to_plain_buffer()))
          this_queue.append(inmessage.to_plain_buffer())
      else:
        pass

    if len(this_queue) > 0:
      self.J1708responses.append((outmessage.to_plain_buffer(),this_queue))

  def j1939_extraction_callback(self,sent_msg,recvd_msg):
    self.J1939responses.append((sent_msg,recvd_msg))


\end{minted}


\TUsubsection{atamessage.py}
\begin{minted}{python}
import copy

READ_REQUEST = 0x70
WRITE_REQUEST = 0x80
READ_WRITE_RESPONSE = 0x90
ENCRYPTED_BROADCAST_REQUEST = 0xA0
ENCRYPTED_BROADCAST_RESPONSE = 0xB0
ENCRYPTED_READ_REQUEST = 0xC0
ENCRYPTED_WRITE_REQUEST = 0xD0
ENCRYPTED_READ_WRITE_RESPONSE = 0xE0
SECURITY_SETUP = 0xF0

__keyarray = [0x19,0xAB,0x5C,0x7D,0xED,0x91,0x96,\
            0x1B,0x8B,0xB7,0xA2,0x78,0x7A,0x89,\
              0x5E,0x0B,0x92,0x4D,0x84]


def is_encrypted_req(num):
  cmd = num & 0xF0
  return cmd >= 0xA0 and cmd <= 0xE0

def is_proprietary(buf):
  return buf[1] == 0xFE

def generate_key(key,code):
  return key + (code % 4)

def decrypt_buf(buf,index):
  if index >= len(__keyarray):
    return buf
  nbuf = [None]*len(buf)
  bl = __keyarray[index]
  for i in range(len(buf)):
    nbuf[i] = buf[i] ^ bl
  return nbuf

def compare_msgs(a,b):
  if not (a.command & 0xF0) == (b.command & 0xF0):
    return False
  for (q,z) in zip(a.data,b.data):
    if not q == z:
      return False

  return True


#encryption and decryption are the same thing
#just creating this helper function for readability
def encrypt_buf(buf,index):
  return decrypt_buf(buf,index)

class ATAMessage():
  def __init__(self,src,dst,command,data,key=None,plain=False):
    self.src = src
    self.dst = dst
    self.command = command
    self.data = copy.deepcopy(data)
    self.key = key
    if is_encrypted_req(command) and not plain and key is not None:
      code = self.command & 0x0F
      tkey = generate_key(key,code)
      self.data = decrypt_buf(data,tkey)
    self.pid = self.data[0]


  @classmethod
  def from_buf(cls, buf,key=None):
    src = buf[0]
    dst = buf[2]
    command = buf[3]
    data = buf[4:]
    return cls(src,dst,command,data,key=key)

  @classmethod
  def from_plain_buf(cls,buf,key=None):
    src = buf[0]
    dst = buf[2]
    if buf[3] >= 0xa0 and buf[3] <= 0xFF:
      command = buf[3] | 0x02
    else:
      command = buf[3]
    data = buf[4:]
    return cls(src,dst,command,data,key=key,plain=True)

  @classmethod
  def from_transport(cls,src,dst,data,key=None):
    command = data[1]
    return cls(src,dst,command,data[2:],key=key)

  def is_encrypted_request(self):
    return is_encrypted_req(self.command)

  def is_setup(self):
    return self.command & 0xF0 == 0xF0

  def _data_out(self,newkey=None):
    if newkey is None:
      thiskey = self.key
    else:
      thiskey = newkey
    if self.is_encrypted_request():
      index = generate_key(thiskey,self.command & 0x0F)
      data = encrypt_buf(self.data,index)
    else:
      data = self.data

    return data

  def to_buffer(self,newkey=None):
    data = self._data_out(newkey)
    return [self.src,0xFE,self.dst,self.command]+data

  def to_plain_buffer(self):
    return [self.src,0xFE,self.dst,self.command&0xF0]+self.data

  def to_transport(self,newkey=None):
    data = self._data_out(newkey)
    return (self.src,self.dst,[0xFE,self.command]+data)






\end{minted}


\TUsubsection{responses.py}
\begin{minted}{python}
#/usr/bin/env python

def compare_J1708_msg(msg1,msg2):
  if not len(msg1) == len(msg2):
    return False
  
  for i in range(0,len(msg1)):
    if (not i == 3 and not msg1[i] == msg2[i]) or (i==3 and not msg1[i] & 0xF0 == msg2[i] & 0xF0):
      return False
  
  return True

def compare_j1939_msg(msg1,msg2):
  print("comparing %s to %s" % (msg1,msg2))
  if not len(msg1) == len(msg2):
    return False

  for i in range(len(msg1)):
    if i == 3:
      pass
    elif i == 1: #and msg1[i] & 0xFE == msg2[i] & 0xFE:
      pass
    elif not (msg1[i] == msg2[i] and not i ==3):
      return False
  return True

#Make this work for both J1939 and J1708
#request: the request to match, should be an array of integers
#request_queue: list of request/response pairs. Each pair is a 2-tuple.
#request_index: the current index in the ResponseQueue for that protocol
#comp_func: function for comparing requests. see compare_* functions above.

def _get_response(request,request_queue,request_index,comp_func):
  if request_index is None:
    return (None,None)

  if comp_func(request,request_queue[request_index][0]):
    return (request_index,request_queue[request_index][1])

  start = request_index
  request_index = (request_index + 1) % len(request_queue)
  while not request_index == start:
    if comp_func(request,request_queue[request_index][0]):
      return (request_index+1,request_queue[request_index][1])
    else:
      request_index = (request_index + 1) % len(request_queue)

  return (start, None)

def _get_multi_response(multicurrent,response_queue):
  for r in response_queue:
    if len(r[1]) <= 1:
      continue
    responses = r[1]
    test_response = responses[0]
    found = True
    for i in range(len(multicurrent)):
      if not test_response[5+i] == multicurrent[i]:
        found = False
        break

    if found:
      return responses

  return None


  
class ResponseQueue():
  def __init__(self, J1708requests = None,J1939requests = None):
    if J1708requests is None:
      self.J1708requests = []
      self.J1708current = None
      self.J1708multicurrent = None
    else:
      self.J1708requests = J1708requests
      self.J1708current = 0

    if J1939requests is None:
      self.J1939requests = []
      self.J1939current = None
    else:
      self.J1939requests = J1939requests
      self.J1939current = 0
  
  def get_j1939(self,request):
    (self.J1939current,response) = _get_response(request,self.J1939requests,self.J1939current,compare_j1939_msg)
    return response

  def get_j1708(self,request):
    if request[4] == 0xD1:
      self.J1708multicurrent = request[5:]
    if request[4] == 0xD3:
      return _get_multi_response(self.J1708multicurrent,self.J1708requests)
    (self.J1708current,response) = _get_response(request,self.J1708requests,self.J1708current,compare_J1708_msg)
    return response
      

\end{minted}
%\lstinputlisting{catreplay/responses.py}

\TUsection{Supporting Code}
\TUsubsection{RP1210Trace.py}\label{app:trace}
\begin{minted}{python}
#!/usr/bin/env python

import ctypes
from pydbg import *
from pydbg.defines import *
import sys
from RP1210Call import *
import utils
import re

RP1210calls = {'RP1210_SendMessage':None,'RP1210_ReadMessage':None,'RP1210_ClientConnect':None,'RP1210_ClientDisconnect':None,'RP1210_SendCommand':None}

#Obtain list of RP1210 DLLs registered on the system
logline = re.compile('APIImplementations=(.*)$')
try:
  rp1210ini = open("C:\\Windows\\RP121032.ini")
except:
  print("Could not open RP121032.ini")
  sys.exit(-1)
liblist = []
for line in rp1210ini.readlines():
  m = re.match(logline,line)
  if m:
    liblist = m.group(1).split(",")

for i in range(len(liblist)):
  liblist[i]+=".dll"



try:
  logfile = open("trace_out_2.txt","w")
except:
  print("Count not open logfile for writing")
  sys.exit(-1)

module_name = None
dbg = pydbg()
dbg.get_debug_privileges()
getproc = dbg.func_resolve("kernel32.dll","GetProcAddress")
procname = sys.argv[1]






#Wait until the desired process starts, then attach to it
attached = False
while not attached:
  for (pid,name) in dbg.enumerate_processes():
    if procname in name:
      dbg.attach(pid)
      print("Attaching to %s, pid %d" % (name,pid))
      attached = True
      break





generic_call_queue = []
read_call_queue = []


hooks = utils.hook_container()

def generic_return_handler(debug,args,ret):
  generic_call_queue[-1].retval = ret
  print(str(generic_call_queue[-1].to_buffer()))
  logfile.write(str(generic_call_queue[-1].to_buffer()))
  logfile.write("\n")
  return DBG_CONTINUE

def smhandler(debug,args):

  clientID = args[0]
  msgbuf = args[1]
  msglen = args[2]
  blockonsend = args[4]

#  debug.bp_set(retholder,handler=generic_return_handler)


  generic_call_queue.append(SendMessage(clientID,msglen,blockonsend,sent_message))
  return DBG_CONTINUE

def rm_return_handler(debug,args,ret):

  try:
    msg = debug.read_process_memory(args[1],ret)
  except:
    msg = "Read Error"

  read_call_queue[-1].retval = ret
  read_call_queue[-1].message = msg
  print(str(read_call_queue[-1].to_buffer()))
  logfile.write(str(read_call_queue[-1].to_buffer()))
  logfile.write("\n")
  return DBG_CONTINUE

def rmhandler(debug, args):

  clientID = args[0]
  bufholder = args[1]
  bufsize = args[2]
  blockonread = args[3]


  read_call_queue.append(ReadMessage(clientID,bufsize,blockonread))
  return DBG_CONTINUE


def cchandler(debug,args):

  device_id = args[1]
  fpchprotocol = args[2]
  txbufsize = args[3]
  rxbufsize = args[4]

  protocol_string = debug.read_process_memory(fpchprotocol,10)
  msg = debug.get_ascii_string(protocol_string)
  if not msg:
    msg = debug.get_unicode_string(protocol_string)

  generic_call_queue.append(ClientConnect(device_id,txbufsize,rxbufsize,msg))
  return DBG_CONTINUE

def cdhandler(debug,args):

  device_id = args[0]


  generic_call_queue.append(ClientDisconnect(device_id))
  return DBG_CONTINUE

def schandler(debug,args):

  commandnumber = args[0]
  clientID = args[1]
  fpchClientCommand = args[2]
  bufsize = args[3]

  commandbuf = debug.read_process_memory(fpchClientCommand,bufsize)

  generic_call_queue.append(SendCommand(commandnumber,clientID,commandbuf,bufsize))
  return DBG_CONTINUE

def get_proc_addr_hook(debug,args,ret):
  try:
    mem = debug.read_process_memory(args[1],40)
  except:
    return DBG_CONTINUE

  msg = debug.get_ascii_string(mem)
  if not msg:
    msg = debug.get_unicode_string(mem)

  if msg in RP1210calls.keys():
    RP1210calls[msg] = ret
    print("%s is at %d" % (msg,ret))
    if not None in RP1210calls.values():
      hooks.add(debug,RP1210calls["RP1210_ReadMessage"],4,rmhandler,rm_return_handler)
      hooks.add(debug,RP1210calls["RP1210_SendMessage"],5,smhandler,generic_return_handler)
      hooks.add(debug,RP1210calls["RP1210_ClientConnect"],6,cchandler,generic_return_handler)
      hooks.add(debug,RP1210calls["RP1210_ClientDisconnect"],1,cdhandler,generic_return_handler)
      hooks.add(debug,RP1210calls["RP1210_SendCommand"],4,schandler,generic_return_handler)
  return DBG_CONTINUE

def load_dll_handler(debug):
  last_dll = debug.get_system_dll(-1)
  #print(last_dll.name)
  if last_dll.name == "kernel32.dll":
    hooks.add(dbg,getproc,2,None,get_proc_addr_hook)

  return DBG_CONTINUE


#Check to see if a RP1210 DLL is loaded already
modules = dbg.enumerate_modules()
loaded = False
rp1210module = None
for module in map(lambda x: x[0],modules):
  if module.lower() in map(lambda x: x.lower(),liblist):
    loaded = True

    rp1210module = module
    break

if loaded:
  print("Found %s, resolving function addresses." % rp1210module)
  sendmsg = dbg.func_resolve_debuggee(rp1210module,"RP1210_SendMessage")
  readmsg = dbg.func_resolve_debuggee(rp1210module,"RP1210_ReadMessage")
  clientconnect = dbg.func_resolve_debuggee(rp1210module,"RP1210_ClientConnect")
  clientdisconnect = dbg.func_resolve_debuggee(rp1210module,"RP1210_ClientDisconnect")
  sendcommand = dbg.func_resolve_debuggee(rp1210module,"RP1210_SendCommand")

  hooks.add(dbg,sendmsg,5,smhandler,generic_return_handler)
  hooks.add(dbg,readmsg,4,rmhandler,rm_return_handler)
  hooks.add(dbg,clientconnect,6,cchandler,generic_return_handler)
  hooks.add(dbg,clientdisconnect,1,cdhandler,generic_return_handler)
  hooks.add(dbg,sendcommand,4,schandler,generic_return_handler)
else:
  print("Couldn't find module in memory, we'll wait for it.")
  dbg.set_callback(LOAD_DLL_DEBUG_EVENT,load_dll_handler)
dbg.run()





\end{minted}

\TUsubsection{J1708 Driver}
\begin{minted}{python}
#!/usr/bin/python3



import serial
import time
from multiprocessing import Process, Pipe, Lock
from functools import reduce
import struct

#TODO: need a thread that monitors the J1708 bus and continually places messages in the receive buffer.
#      upon initialization, thread needs to spawn and synchronize itself with the J1708 bus.

TENBITTIMES = .0010417

def toSignedChar(num):
  if type(num) is bytes:
    return struct.unpack('b',num)[0]
  else:
    return struct.unpack('b',struct.pack('B',num & 0xFF))[0]

def checksum(msg):
  if type(msg[0]) is bytes:
    thismsg = list(map(lambda x: int.from_bytes(x,byteorder='big'),msg))
  else:
    thismsg = msg
  return toSignedChar(~reduce(lambda x,y: (x + y) & 0xFF, list(thismsg)) + 1)

def check(msg):
  if type(msg[0]) is bytes:
    thismsg = list(map(lambda x: int.from_bytes(x,byteorder='big'), msg))
  else:
    thismsg = msg
  return toSignedChar(reduce(lambda x,y: (x+y) & 0xFF, list(thismsg)))

def initialize(busport,buslock):
  synced = False
  buslock.acquire()
  qtime = time.time()
  busport.flushInput()
  busport.timeout = 0
  while not synced:
    #print(time.clock())
    a = busport.read(1)
    if not a is b'':
      qtime = time.time()
    elif time.time() - qtime < TENBITTIMES:
      continue
    else:
      synced = True

  buslock.release()




    
def getmsg(busport,buslock):
  finished = False
  msg = []
  buslock.acquire()
  ttimeout = busport.timeout
  busport.timeout = TENBITTIMES
  tempb = b''
        #periodically timeout and let a blocking sender go.
  while tempb is b'':
    tempb = busport.read(1)
    if tempb is b'':
      buslock.release()
      buslock.acquire()
  msg += [tempb]
  busport.timeout = 0
  stime = time.time()
  
  while not finished:
    a = busport.read(1)
    t = time.time()
    if a is b'' and t - stime > TENBITTIMES:
      finished = True
    elif a is b'':
      continue
    else:
      stime = t
      msg += [a]

  busport.timeout=ttimeout
  buslock.release()
  if not check(msg[:-1]) + toSignedChar(msg[-1]) == 0:
    return None
  else:
    return msg

    #msgqueue: a list type to enqueue messages
    #queuelock: a lock object that controls access to the queue
def run(busport,buslock,p):
  initialize(busport,buslock)
  while True:
    thismsg = getmsg(busport,buslock)
    if thismsg is not None:
      p.send(thismsg)
#test
#    def __init__(self,busport=None,buslock=None,msglock=None,msgqueue=None):
class J1708():
  def __init__(self,uart=None):
    self._sport = None
    if uart is not None:
      self._sport = serial.Serial(port=uart,baudrate=9600,
                bytesize=serial.EIGHTBITS,
                stopbits=serial.STOPBITS_ONE)

    self.buslock = Lock()
    self.mypipe, self.otherpipe = Pipe()
    self.r_proc = Process(target=run, args=(self._sport,self.buslock,self.otherpipe))
    self.r_proc.start()


  def read_message(self,timeout=None):
    if not self.mypipe.poll(timeout):
      return None

    retval = self.mypipe.recv()

    return retval

    #currently relying on the read_thread to maintain synchronization
  def send_message(self,msg):
    retval = 0
    thismsg = bytes(msg)
    chksum = struct.pack('b',checksum(thismsg))
    thismsg += chksum
    with self.buslock:
      retval = self._sport.write(thismsg)
      self._sport.flushInput()#solve "echo" problem
    return retval
     
  def __del__(self):
    self.r_proc.terminate()


if __name__ == "__main__":
  thisport = J1708("/dev/ttyO2")
  
  count = 0
  while count < 50:
    a = thisport.read_message()
    if a is not None:
      print(a)
    count += 1

  thisport.send_message([0xac,0xfe,0x80,0xf0,0x17])#SecuritySetup message for CAT ECM.
                                                   #If all is well, will see a response setup
  count = 0
  while count < 50:
    a = thisport.read_message()
    if a is not None:
      print(a)
    count += 1

  del(thisport)
\end{minted}

\TUappendix{Hardware Schematic}\label{app:schematic}

\includepdf[pages={1-4}]{HardwareSchematic.pdf}

%\end{appendices}
