\chapter{Conclusions and Future Work}

\section{Contributions}

The work detailed in this dissertation makes the following contributions:

\begin{itemize}

  \item This dissertation is the first attempt to address the digital forensic concerns raised in the
        paper by Johnson, et al \cite{Johnson2014}. As such, it provides a framework by which future
        forensic solutions may be evaluated.
  \item The forensic record-replay mechanism is a forensically sound imaging technique for heavy truck ECMs.
        No similar solution exists in the literature, other than physically opening an ECM and reading the
        memory physically.
  \item The cryptosystem provides a general method for preventing malicious tampering with extracted evidence,
        no matter what evidence is being extracted by what device. This has implications for many fields other
        than ECM extractions.

\end{itemize}

\section{Summary of Provided Solution}

As can be seen from the diffs of Warranty Files from the various extractions, replayed data is consistent with
traditionally extracted data in all respects, and is in fact more consistent in some areas.

\subsection{Forensic Soundness}

As can be seen from the verification results, the original evidence is modified less using forensic replay than it is by repeatedly
extracting the information by traditional means. Even if no additional faults are created during a bench download, the ECM running
time is better preserved by imaging and replaying the ECM data rather than repeated downloads.

The expertise requirement is not totally alleviated by the extraction and replay process; an investigator still has to know how to 
connect to the ECM and power it up properly. However, the main advantage of the forensic extraction and replay process is that
all information extraction is automated; no knowledge of diagnostic software or the steps required to gather pertinent crash information
is needed. This is an advantage in a law enforcement context where training time is at a premium.

By default, the CAT ECM extraction process is almost completely opaque. Unless certain data are not available on the ECM, or an error
occurs during the download process, there is no record kept of the traffic other than its final interpretation by the maintenance
software. As the forensic replay method is literally a recording of network traffic, this traffic can be examined after the fact
to verify the extraction and replay process.

Eoghan Casey's definition of forensic soundness includes seven levels of forensic certainty based on certain characteristics
of the evidence and the method in which it is stored:

As to Eoghan Casey's definition of forensic soundness, he solution outlined herein makes no claim about the veracity of the evidence as it relates to physical phenomena, except that it 
is a faithful representation of the information communicated
by the ECM, but the other characteristics of evidence can be compared. The main difference here is that when extracting and storing evidence through traditional means, even if the
evidence agrees with other sources it is not tamper-proof, and thus can only achieve a rating of C4 in the absolute best case (i.e. there is physical evidence corroborating the 
digital evidence). However, with the tamper protections added by the cryptosystem presented herein, the digital evidence is rated at C4 at bare minimum, reaching C5 or C6 with corroborating
physical evidence. 

\subsection{Practicality}

Currently, in order to perform an evidence download from a heavy vehicle ECM, investigators require at bare minimum a laptop and a diagnostic link connector.
If a bench download is to be performed, an engine flashing harness is required as well. Unfortunately, removing the need for an engine flashing harness is
beyond the scope of this work. However, this replay solution represents a logistical advantage over existing methods.

While the current use case still requires a laptop, it does not require any software beyond a simple web browser. This greatly simplifies software logistics, as well
as reducing the cost of software required to conduct investigations in the field. The hardware can be made no larger or heavier than existing diagnostic link connectors
while remaining far cheaper.

\section{Future Work}

\subsection{Understanding Meaning}

The current solution requires that investigators use the manufacturers' maintenance software to interpret the data that has been extracted and replayed from the ECM.
While this is an improvement over the current situation in terms of ease and practicality, it is also not ideal. Future work in this area should include analyzing the
meaning of the extracted ECM data, so that it can be analyzed and presented independently of proprietary software.

\subsection{Information Verification}

New work might also include verification of the interpretation of the data. Currently, it is assumed that the data reported by the ECM accurately reflect the conditions
observed during incidents, and that the manufacturers' software accurately interprets this reported data. 

% LocalWords:  ics
