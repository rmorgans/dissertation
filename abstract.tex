Since the early 1990s, the trucks' Engine Control Modules (ECMs) have recorded information valuable to accident
reconstructionists. This information is extracted using the engine manufacturers' maintenance software in
a manner that does not protect the evidence from alteration. This dissertation describes a novel method
of extracting, and replaying, information from heavy truck engine control modules. This method preserves
the integrity of the original evidence, is faithful to the original evidence source, and is cryptographically
protected. The extraction/replay methodology combines a generic method with extensions specific to
the manufacturer's proprietary protocols, which were developed by reverse-engineering manufacturer software.
A cryptosystem is also described that protects the information from modification, whether accidental or malicious.
It was found that the replay method generated the same report data as the ECM it recorded. It was also found that
multiple extractions changed the source evidence, indicating the forensic superiority of the replay method.
The replay method fulfills the criteria of forensic soundness and addresses some problems with current
evidence handling procedures.
