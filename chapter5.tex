\TUchapter{Cryptographic Protection}

To maximally ensure the data extracted are protected from alteration,
they needed some form of cryptographic protection that would make it impossible, or at
least very unlikely, that the evidence could be altered in a manner that would not be
detected. The protection requirements are as follows:

\begin{enumerate}
  \item The data must be protected from alteration
  \item The data must be protected despite the fact that all computation takes
        place on a device that is solely within the control of an unknown person.
  \item Others' data must be secure even if a single device is compromised.

\end{enumerate}

\TUsection{Protection Mechanism}

In traditional computer forensic investigations, a disk image is protected by performing
a cryptographic hash on it. Later on, the image is hashed again and the two hashes are compared
to confirm that the image has not changed. The disk imaging process frequently takes place in an
office, where all parties involved in the case can easily be present to ensure that the extraction
is performed properly and no data are altered.

In the case of ECM data records, however, the use case is somewhat different. The data are frequently
extracted in remote locations where it is not feasible to have all parties to the case present. Therefore,
the individual extracting the information has total access to the information being extracted, likely for
a significant length of time. In this case, hashing alone may not offer the required protection as the data
may just be altered and the hash recomputed. While this is also a risk in hard drive extractions, the possibility
of having all parties present mitigates that somewhat.

Rather than attempting to protect the data from alteration, which is practically impossible with the device
in the physical control of a potentially malicious actor, the solution is to strongly encrypt
the data instead. If the data are strongly encrypted, while altering the data may be possible, 
\emph{meaningfully} altering the data is almost impossible. By ensuring that an attacker will gain nothing by
altering the data, the data are effectively prevented from being altered.

\TUsection{Cryptosystem}

While strongly encrypting the data protects it from alteration, encrypting the data on a remote device in such
a way that the key is not discovered is a non-trivial exercise. The following encryption algorithm was developed
to perform this task:

\begin{enumerate}
  \item A nonce, to be used as a symmetric key, is randomly generated.
  \item A second nonce, to be used as an initialization vector (IV) for the symmetric cipher is generated.
  \item The nonce is used to encrypt the data.
  \item A public key, stored on the device, is used to encrypt the key.
  \item The encrypted key is stored alongside the encrypted disk image.
  \item Later, the RSA private key, stored with a trusted third party, is
        used to decrypt the symmetric key, which is then used to decrypt
        the data.
\end{enumerate}

This is an example of a hybrid cryptosystem as described in \cite{cramer2004}. Cramer and Shoup prove that a hybrid cryptosystem of
this type is secure so long as the underlying algorithms are secure, and the padding scheme used for encrypting the key is secure.

\TUsection{Cryptosystem Implementation}

While the general cryptosystem is agnostic in terms of the actual algorithms used, a practical implementation requires specific
algorithms to be chosen.

The symmetric algorithm chosen to protect the data is AES-128, as the AES algorithm is the industry standard symmetric encryption
algorithm, and it is currently believed to be secure. The 128-bit key length was chosen because of breaks discovered in the 256-bit
key length\cite{Biryukov2009}. As is standard cryptographic practice \cite{NISTSP80038A}, the AES cipher is run in the Cipher Feedback (CFB)
operating mode to prevent frequency analysis attacks.

The cryptographic hash function chosen is SHA-256. While a longer hash value may yield better security, a longer hash may also make
it more difficult to write down a hash value for an investigator in the field. SHA-2 was chosen over SHA-1 because of the widely-published
attacks on SHA-1\cite{Manuel2011}.

In keeping with current security best practices, the symmetric keys are padded according to the PKCS\#1-OAEP standard before encryption\cite{Bellare1995}.

\TUsubsection{Key Generation}

As all of the components of this cryptosystem are standard cryptographic primitives in common use in industry today, for the purposes of this
dissertation they may safely be assumed to be secure. The security of the system hinges on the keys used by the algorithms. The public/private
keypairs are generated on a workstation prior to the deployment of the system, using industry-standard software, so those are as secure
as is feasible.

Symmetric keys, however, are generated on-the-fly using the pseudorandom number generator (PRNG) provided by the Linux kernel running
on the embedded device itself. Thus, the security of the system hinges entirely on the security of the Linux PRNG. In order to be sure of
the security of the key generation scheme, it was necessary to verify the security of the PRNG as implemented on this particular embedded ARM
system.

First, a review of the concepts of a PRNG may be helpful. A more in-depth analysis may be found in Gutterman's 2006 paper\cite{Gutterman2006}
as well as the NIST tech report SP800-90A, Recommendation for Random Number Generation Using Deterministic Random Bit Generators\cite{NISTSP80090A}.
By definition, a PRNG is a deterministic bit generator that takes as its input a secret initial value, called a seed\cite{NISTSP80090A}.
For a continuously running PRNG such as one provided by the Linux kernel, the PRNG may be continuously seeded with true random input, termed entropy input.
There are several facets of security of PRNGs, two of which Gutterman terms ``forward security'' and ``backward security''\cite{Gutterman2006}.
A forward-secure PRNG has the property that if an adversary learns the state of the generator at a certain time, she cannot deduce previous
states of the generator. Conversely, a backward-secure PRNG has the property that an attacker cannot deduce future states of the generator
if she learns the internal state of the generator at a certain time, provided that sufficient entropy is used to refresh the generator state.

The Linux kernel provides a cryptographically secure random number generator, which provides two devices to userspace, \texttt{/dev/random} and
\texttt{/dev/urandom}\cite{goichon2012} .  Both devices are fed from the same entropy pool, which has a counter associated with it to estimate the amount of
entropy in the system. The difference between the two is that reads to \texttt{/dev/random} will block if the estimated level of entropy drops below
a threshold value, while \texttt{/dev/urandom} will not. Entropy is replenished by three kernel functions that are exported to device drivers: they are called
\texttt{add\_disk\_randomness()}, \texttt{add\_input\_randomness()}, and \texttt{add\_interrupt\_randomness()}.

In 2006, Gutterman et al \cite{Gutterman2006} performed an in-depth analysis of the security of the random number generator provided by the Linux operating system. 
Of particular interest was their analysis of the Linux PRNG on routers running the OpenWRT distribution. As a PRNG is a deterministic algorithm, its security 
relies upon the randomness of its entropy input. In practice, Linux systems use physical phenomena such as disk seek times, and user input
such as mouse and keyboard use as their source of entropy inputs. However, embedded systems such as OpenWRT routers use solid-state disks
with deterministic access times, and they lack user input from keyboards and mice. The authors were able to leverage this lack of entropy
input to successfully break the backward secrecy of OpenWRT routers.

As this cryptosystem would run on a very similar system, with even less entropy input than a router (routers typically use receipt of network
traffic as an entropy source), the backward security of the Linux PRNG was a grave concern. Fortunately, the TI Sitara AM3358 System on a Chip (SOC)
on which the BeagleBone is based includes an onboard hardware random number genrator (HWRNG) that produces true random output. Unfortunately, the
kernel version which includes support for the various hardware needed for vehicle network communication did not include support for the AM33XX
hardware random number generator. This situation was resolved by adapting driver code from the TI software development kit, based on Linux
kernel version 3.12, to the 3.8 kernel version. This modification is now included in the official BeagleBone kernel source.

This new source of entropy was leveraged using the rng-tools software suite. Instead of using random data from the HWRNG directly, rng-tools
reads data from the HWRNG, applies a test to ensure its randomness, and then feeds the Linux kernel's entropy pool with this data.

\TUsection{Proposed Integration Into Replay System}

While the specification of the cryptosystem itself only specifies that the corresponding private key be held by a trusted third party, it does
not specify how the cryptosystem is integrated into the use of this device. While the manner in which it is integrated is up to the ultimate
user of the system, this section presents some proposed usage scenarios.

\TUsubsection{Central Decrypting Station}

In this use case, security keys are held by a trusted 3rd party. For example, the device manufacturer or an escrow service can act as
trusted third parties. When the data are to be decrypted, the encrypted file container may be sent to the trusted third party, decrypted,
and then sent to all parties of the case individually over standard encrypted channels. The original cryptographic hashes generated
during the extraction may be used to verify integrity of the unencrypted file after its return.

\TUsubsection{USB Key Tokens}

While the previous use case requires a network connection, it is possible to implement a solution that doesn't require network communications.
For example, the private key may be sent to a local trusted third party (such as a judge). The key can then reside on a USB token until the
replay takes place. In this case, the decrypted replay can take place in full view of all parties to the case in a controlled environment.

