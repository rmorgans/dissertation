\chapter{Cryptographic Protection}

To maximally ensure the data extracted are protected from alteration,
they needed some form of cryptographic protection that would make it impossible, or at
least very unlikely, that the evidence could be altered in a manner that would not be
detected. The protection requirements are as follows:

\begin{enumerate}
  \item The data must be protected from alteration
  \item The data must be protected despite the fact that all computation takes
        place on a device that is solely within the control of an unknown person.
  \item Others' data must be secure even if a single device is compromised.

\end{enumerate}

\section{Protection Mechanism}

In traditional computer forensic investigations, a disk image is protected by performing
a cryptographic hash on it. Later on, the image is hashed again and the two hashes are compared
to confirm that the image has not changed. The disk imaging process frequently takes place in an
office, where all parties involved in the case can easily be present to ensure that the extraction
is performed properly and no data are altered.

In the case of ECM data records, however, the use case is somewhat different. The data are frequently
extracted in remote locations where it is not feasible to have all parties to the case present. Therefore,
the individual extracting the information has total access to the information being extracted, likely for
a significant length of time. In this case, hashing alone may not offer the required protection as the data
may just be altered and the hash recomputed. While this is also a risk in hard drive extractions, the possibility
of having all parties present mitigates that somewhat.

Rather than attempting to protect the data from alteration, which is practically impossible with the device
in the physical control of a potentially malicious actor, the solution is to strongly encrypt
the data instead. If the data are strongly encrypted, while altering the data may be possible, 
\emph{meaningfully} altering the data is impossible. By ensuring that an attacker will gain nothing by
altering the data, the data are effectively prevented from being altered.

\section{Cryptosystem}

While strongly encrypting the data protects it from alteration, encrypting the data on a remote device in such
a way that the key is not discovered is a non-trivial exercise. The following encryption algorithm was developed
to perform this task:

\begin{enumerate}
  \item A nonce, to be used as a symmetric key, is randomly generated.
  \item The nonce is used to encrypt the data.
  \item A public key, stored on the device, is used to encrypt the key.
  \item The encrypted key is stored alongside the encrypted disk image.
  \item Later, the RSA private key, stored with a trusted third party, is
        used to decrypt the symmetric key, which is then used to decrypt
        the data.
\end{enumerate}

This is an example of a hybrid cryptosystem as described in \cite{cramer2004}. Cramer and Shoup prove that a hybrid cryptosystem of
this type is secure so long as the underlying algorithms are secure, and the padding scheme used for encrypting the key is secure.

\section{Cryptosystem Implementation}

While the general cryptosystem is agnostic in terms of the actual algorithms used, a practical implementation requires specific
algorithms to be chosen.

The symmetric algorithm chosen to protect the data is AES-128, as the AES algorithm is the industry standard symmetric encryption
algorithm, and it is currently believed to be secure. The 128-bit key length was chosen because of breaks discovered in the 256-bit
key length\cite{Biryukov2009}.

The cryptographic hash function chosen is SHA-256. While a longer hash value may yield better security, a longer hash may also make
it more difficult to write down a hash value for an investigator in the field. SHA-2 was chosen over SHA-1 because of the widely-published
attacks on SHA-1\cite{Manuel2011}.

In keeping with current security best practices, the symmetric keys are padded according to the PKCS\#1-OAEP standard before encryption\cite{Bellare1995}.

